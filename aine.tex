% --- Template for thesis / report with tktltiki2 class ---
% 
% last updated 2013/02/15 for tkltiki2 v1.02

\documentclass[finnish]{tktltiki2}

% tktltiki2 automatically loads babel, so you can simply
% give the language parameter (e.g. finnish, swedish, english, british) as
% a parameter for the class: \documentclass[finnish]{tktltiki2}.
% The information on title and abstract is generated automatically depending on
% the language, see below if you need to change any of these manually.
% 
% Class options:
% - grading                 -- Print labels for grading information on the front page.
% - disablelastpagecounter  -- Disables the automatic generation of page number information
%                              in the abstract. See also \numberofpagesinformation{} command below.
%
% The class also respects the following options of article class:
%   10pt, 11pt, 12pt, final, draft, oneside, twoside,
%   openright, openany, onecolumn, twocolumn, leqno, fleqn
%
% The default font size is 11pt. The paper size used is A4, other sizes are not supported.
%
% rubber: module pdftex

% --- General packages ---

\usepackage[utf8]{inputenc}
\usepackage[T1]{fontenc}
\usepackage{lmodern}
\usepackage{microtype}
\usepackage{amsfonts,amsmath,amssymb,amsthm,booktabs,color,enumitem,graphicx}
\usepackage[pdftex,hidelinks]{hyperref}
\usepackage[ruled,vlined,linesnumbered]{algorithm2e}
\usepackage{caption}

% Algorithm2e environment with "Algoritmi"-caption.
\newenvironment{finalgo}[1][htb]{
  \renewcommand{\algorithmcfname}{Algoritmi}
  \begin{algorithm}[#1]
}{\end{algorithm}}

% To be able to not numbering individual lines:
\let\oldnl\nl% Store \nl in \oldnl
\newcommand{\nonl}{\renewcommand{\nl}{\let\nl\oldnl}}

% Automatically set the PDF metadata fields
\makeatletter
\AtBeginDocument{\hypersetup{pdftitle = {\@title}, pdfauthor = {\@author}}}
\makeatother

% --- Language-related settings ---
%
% these should be modified according to your language

% babelbib for non-english bibliography using bibtex
\usepackage[fixlanguage]{babelbib}
\selectbiblanguage{finnish}

% add bibliography to the table of contents
\usepackage[nottoc]{tocbibind}
% tocbibind renames the bibliography, use the following to change it back
\settocbibname{Lähteet}

% --- Theorem environment definitions ---

\newtheorem{lau}{Lause}
\newtheorem{lem}[lau]{Lemma}
\newtheorem{kor}[lau]{Korollaari}

\theoremstyle{definition}
\newtheorem{maar}[lau]{Määritelmä}
\newtheorem{ong}{Ongelma}
\newtheorem{alg}[lau]{Algoritmi}
\newtheorem{esim}[lau]{Esimerkki}

\theoremstyle{remark}
\newtheorem*{huom}{Huomautus}

\DeclareMathOperator*{\argmin}{arg\, min}

% --- tktltiki2 options ---
%
% The following commands define the information used to generate title and
% abstract pages. The following entries should be always specified:

\title{Polunhakualgoritmit ja -järjestelmät}
\author{Rodion Efremov}
\date{\today}
\level{Kandidaatintutkielma-aine}
\abstract{Tiivistelmä.}

% The following can be used to specify keywords and classification of the paper:

\keywords{a, bb, ccc}

% classification according to ACM Computing Classification System (http://www.acm.org/about/class/)
% This is probably mostly relevant for computer scientists
% uncomment the following; contents of \classification will be printed under the abstract with a title
% "ACM Computing Classification System (CCS):"
% \classification{}

% If the automatic page number counting is not working as desired in your case,
% uncomment the following to manually set the number of pages displayed in the abstract page:
%
% \numberofpagesinformation{16 sivua + 10 sivua liitteissä}
%
% If you are not a computer scientist, you will want to uncomment the following by hand and specify
% your department, faculty and subject by hand:
%
% \faculty{Matemaattis-luonnontieteellinen}
% \department{Tietojenkäsittelytieteen laitos}
% \subject{Tietojenkäsittelytiede}
%
% If you are not from the University of Helsinki, then you will most likely want to set these also:
%
% \university{Helsingin Yliopisto}
% \universitylong{HELSINGIN YLIOPISTO --- HELSINGFORS UNIVERSITET --- UNIVERSITY OF HELSINKI} % displayed on the top of the abstract page
% \city{Helsinki}
%


\begin{document}

% --- Front matter ---

\frontmatter      % roman page numbering for front matter

\maketitle        % title page
\makeabstract     % abstract page

\tableofcontents  % table of contents

% --- Main matter ---

\mainmatter       % clear page, start arabic page numbering

\section{Johdanto}
Polunhaku painotetuissa tai painottamattomissa verkoissa on perustavanlaatuinen ongelma, joka ei ole mielenkiintoinen vain itsessään, vaan on toisinaan tarvittava alioperaatio muissa algoritmeissa. Esimerkiksi Edmond-Karpin algoritmi käyttää leveyssuuntaisen haun ratkaistaessaan maksimivuo-ongelmaa; multiple sequence alignment -ongelmaa on ruvettu viime vuosikymmeninä ratkomaan myös heuristisin polunhakualgoritmein.

Verkoista puhuttaessa verkko $G$ on kaksikko $(V, A)$, jossa $V$ on solmujen joukko, ja $A \subset V \times V$ on (suunnattujen) kaarien joukko. Suuntaamaton verkko $G' = (V, E)$ voidaan aina simuloida suunnatulla verkolla $G= (V, A)$ siten, että jokaista suuntaamatonta kaarta $\{ u, v \} \in E$ kohti laitetaan $A$:han kaaret $(u, v)$ ja $(v, u)$. (Suunnattu verkko on suuntamattoman yleistys.) Polunhakua varten, verkosta erotellaan kaksi solmua: lähtösolmu $s$ ja maalisolmu $t$. Jatkossa, $n = |V|$ ja $m = |E|$; näin esimerkiksi leveyssuuntaisen haun aikavaativuus on $O(n + m)$. Polku on $\gamma_k = \langle u_0, u_1, \dots, u_k \rangle$, missä mikään solmu ei esiinny yhtä kertaa enempää, ja verkossa on kaari $(u_i, u_{i + 1})$ jokaisella $i = 0, 1, \dots, k - 1$. Polkuun liittyvä kustannus on sen kaarien painojen summa, ja mitä tulee itse painoihin, ne oletetaan olevan ei-negatiivisia. Ei-painotettujen verkojen kohdalla, jokaisen kaaren paino oletetaan olevan 1.

% --- References ---
%
% bibtex is used to generate the bibliography. The babplain style
% will generate numeric references (e.g. [1]) appropriate for theoretical
% computer science. If you need alphanumeric references (e.g [Tur90]), use
%
% \bibliographystyle{babalpha-lf}
%
% instead.
\section{Tavallisimmat algoritmit}
Edsger W. Dijkstra esitti vuonna 1959 kuuluisan polunhakualgoritminsa, joka käy polynomisessa ajassa \cite{Dijkstra59}. Algoritmi voidaan pitää yhdistävän ``ahneuden'' (engl. \textit{greedy algorithm}), dynaamisen ohjelmoinnin ja inkrementaalisen lähestymistavan. Saatuaan lähtösolmun $s$, algoritmi laskee lyhimpien polkujen puun lähtien solmusta $s$ kunnes $t$ joutuu \textit{avoimeen listaan} (engl. \textit{open list; search frontier}), ja sitä kautta \textit{suljettuun listaan} (engl. \textit{closed list; settled node list}), jolloin lyhin $s, t$-polku on löytynyt. Hart et al. esittivät vuonna 1968 kuuluisan $A\ast$-algoritminsa,  joka -- samoin kuten Djikstran algoritmi -- ylläpitää mm. kunkin saavutetun solmun $u$ $g$-arvon $g(u)$, joka on toistaiseksi pienin kustannus lähtösolmusta $s$ solmuun $u$, ja joka on taattu olemaan pienin mahdollinen heti kun $u$ poistuu avoimesta listasta \cite{Hart68}. Erona on kuitenkin se, että $A\ast$ käyttää kunkin solmun $u$ prioriteettinä sen $f$-arvo, joka on siis $f(u) = g(u) + h(u)$, jossa $h(u)$ on solmun $u$ optimistinen (eli aliarvioitu) etäisyys maalisolmuun. Intuitio tämän järjestelyn takana on se, että $A\ast$ ``tietää'' mihin suuntaan haku on suunnattava, jota pääsisi maalisolmuun, ainakin paremmin kuin Dijkstran algoritmi, jonka hakuavaruus kasvaa laajenevan pallon tavoin ``kaikkiin suuntiin''.
\begin{finalgo}
\nonl Monikkosijoitus \\
$\text{OPEN}, \text{CLOSED}, g, \pi = (\{ s \}, \emptyset, \{  (s, 0) \}, \{ (s, \textbf{nil}) \})$ \\
\While{$|\text{OPEN}| > 0$}{
 $u = \underset{x \in \text{OPEN}}{\argmin}\, g(x)$ \\
 $\text{OPEN} = \text{OPEN} - \{ u \}$ \\
 \If{$x \textbf{\emph{ is }} t$}{
   \KwRet \textsc{Traceback-Path$(t, \pi, \textbf{nil})$}\\
 }
 CLOSED = CLOSED $\cup \{ x \}$ \\
 \nonl Jokaisella solmun $x$ lapsisolmulla $u$, tee... \\
 \For{$(x, u) \in G.A$}{
   \If{$u \in $ CLOSED}{
     $\textbf{continue}$ \\
   }
   $g' = g(x) + w(x, u)$ \\
   \uIf{$u \not \in $ OPEN}{
     OPEN $=$ OPEN $\cup \{ u \}$ \\
     $g(u) = g'$ \\
     $\pi(u) = x$ \\
   }
   \uElseIf{$g(u) > g'$}{
     $g(u) = g'$ \\
     $\pi(u) = x$ \\
   }
 }
}
\nonl Ei $s, t$ -polkua verkossa $G$. \\
\KwRet $\langle \rangle$ \\
\caption{\textsc{Dijkstra-Shortest-Path}$(G, s, t, w)$}
\label{Dijkstra}
\end{finalgo}
$A\ast$:n pseudokoodi on tasan sama kuin Dijkstran algoritmin (Algoritmi \ref{Dijkstra} sivulla \pageref{Dijkstra}), paitsi että rivillä 6 $g(x)$:n sijasta on $f(x)$, jolle siis $f(x) = g(x) + h(x)$.
\begin{finalgo}
$u = x$ \\
$p = \langle  \rangle$ \\
\While{$u \textbf{\emph{ is not nil}}$}{
lisää $u$ $p$:n alkuun \\
$u = \pi(u)$
}
\nonl Kaksisuuntainen haku? \\
\If{$\pi_{REV} \textbf{\emph{ is not nil}}$}{
  $u = \pi_{REV}(x)$ \\
  \While{$u \textbf{\emph{ is not nil}}$}{
    lisää $u$ $p$:n loppuun \\
    $u = \pi_{REV}(u)$ \\
  }
}
\KwRet $p$ \\
\caption{\textsc{Traceback-Path}$(x, \pi, \pi_{REV})$}
\end{finalgo}

\section{Kaksisuuntainen haku}
Vaikka $A\ast$ on tyypillisesti tehokkaampi kuin Dijkstran algoritmi, käyttämällä \textit{kaksisuuntaista} hakua, voidaan päästää verrattavissa olevaan suorituskykyyn. Ajatus kaksisuuntaisuuden takana on se, että algoritmi kasvattaa kaksi hakupuuta, yksi normaaliin tapaan ja toinen maalisolmusta ihan kuin kaaret olisivat ``käännetty'' päinvastaiseen suuntaan, kunnes kaksi hakuavaruutta ``kohtaavat'' keskellä. Nyt jos lyhin polku koostuu $N$ kaaresta, ja verkon solmujen keskiarvoinen aste on $d$, tavallinen, eli yksisuuntainen haku tekee työn
\[
\sum_{i = 0}^N d^i,
\]
kun kaksisuuntainen olisi tehnyt vain
\[
2 \sum_{i = 0}^{\lceil N / 2 \rceil} d^i.
\]
Ylläoleva pätee leveyssuuntaiseen hakuun sellaisenaan, ja painotetun haun kohdalla voidaan saada yläraja kertomalla kunkin summan termin tekijällä $O(\log n)$. 
\subsection{Kaksisuuntainen Dijkstran algoritmi}

\begin{finalgo}
$\text{OPEN}, \text{CLOSED}, g, \pi = \{ s \}, \emptyset, \{ (s, 0) \}, \{ (s, \textbf{nil}) \}$ \\
$\text{OPEN}_{REV}, \text{CLOSED}_{REV}, g_{REV}, \pi_{REV} = \{ t \}, \emptyset, \{ (t, 0) \}, \{ (t, \textbf{nil}) \}$ \\
$\mu = \infty$ \\
$m = \textbf{nil}$ \\ 
\While{$|\text{OPEN}| \cdot |\text{OPEN}_{REV}| > 0$}{
  \If{$m$ \textbf{\emph{is not nil}}}{
    $p = $ \textsc{Check-Bi-Dijkstra-Termination}$f_1$ \\
    \If{$p$ \textbf{\emph{is not nil}}}{
    	\KwRet $p$
    }
  }
  \nonl Triviaali kuormantasaus \\
  \If{$|\text{OPEN}| < |\text{OPEN}_{REV}|$}{
    $u = \underset{x \in \text{OPEN}}{\argmin } \, g(x)$ \\
    $\text{OPEN} = \text{OPEN} - \{ u \} $ \\
    $\text{CLOSED} = \text{CLOSED} \cup \{ u \} $ \\
    \For{$(u, x) \in G.A$}{
      \If{$x$ \textbf{\emph{is in }} $\text{CLOSED}$}{
        \textbf{continue} \\
      }
    }
  }
  \Else{
    $u = \underset{x \in \text{OPEN}_{REV}}{\argmin} g_{REV}(x)$ \\
    $\text{OPEN}_{REV} = \text{OPEN}_{REV} - \{ u \} $ \\
    $\text{CLOSED}_{REV} = \text{CLOSED}_{REV} \cup \{ u \} $ \\
    \nonl Jokaiselle solmun $u$ vanhempainsolmulle $x$, tee.. \\
    \For{$(x, u) \in G.A$}{
      \If{$x$ \textbf{\emph{is in }} $\text{CLOSED}_{REV}$}{
        \textbf{continue} \\
      }
    }
  }
}
\KwRet $\langle  \rangle$ \\
\caption{\textsc{Bidirectional-Dijkstra-Shortest-Path}$(G, s, t, w)$}
\end{finalgo}

\subsection{Kaksisuuntainen $A\ast$}

\section{Prioriteettijonon valinta}
Polkua hakiessa painotetuissa verkoissa joudutaan käyttäämään prioriteettijonoja, jotka ovat tarpeellisia pitääkseen haut optimaaleina, ja joiden oletetaan tarjoavan ainakin neljä operaatiota:
\begin{enumerate}
\item \textsc{Insert}$(H, x, k)$ tallettaakseen solmun $x$ sen prioriteetin $k$ kera,
\item \textsc{Decrease-Key}$(H, x, k)$ päivittääkseen solmun $x$ talletetun prioriteetin (pienemmäksi),
\item \textsc{Extract-Minimum}$(H)$ poistaakseen pienimmän prioriteetin omaava solmu, ja
\item \textsc{Is-Empty}$(H)$ varmistaakseen, että jonossa on vielä alkioita.
\end{enumerate}
Helpoin tehokkaaksi kutsuttu prioriteettijonorakenne (jatkossa vain ``keko'') on binäärikeko, jonka operaatiot (1) - (3) käyvät ajassa $O(\log n)$, jolloin tällaisella keolla Dijkstran ja $A\ast$-algoritmit käyvät kumpikin ajassa $O((m + n) \log n)$. Teoriassa edelläoleva ylläraja voidaan parantaa käyttämällä Fibonacci-kekoa, jonka lisäysoperaatio (1) käy eksaktissa vakioajassa, päivitysoperaatio (2) tasoitetussa vakioajassa, ja poisto-operaatio (3) tasoitetussa ajassa $O(\log n)$, jolloin haut voidaan suorittaa ajassa $O(m + n \log n)$. Huomaa, että kaikki tähän asti mainitut keot perustuvat vertailuihin, ja teoriassa enintään yksi operaatiosta \textsc{Insert} tai \textsc{Extract-Minimum} voi käydä (eksaktissa tai tasoitetussa) vakioajassa, ja toisen on käyttävä ajassa $\Omega(\log n)$, koska muuten algoritmi \ref{genheapsort} tällaisella keolla rikkoisi lajittelemisen informaatioteoreettisen rajan, joka on $\Omega(n \log n)$.
\begin{finalgo}
$H = \varnothing: \text{tyhjennä keko.}$ \\
\For{i = 1 \emph{\KwTo} |S|}{
  \textsc{Insert}$(H, S[i], S[i])$ : $S[i]$ on itsensä prioriteetti. \\
}
\For{i = 1 \emph{\KwTo} |S|}{
  $S[i] = $\textsc{Extract-Minimum}$(H)$  \\
}
\caption{\textsc{Generic-Heap-Sort}($S, H$)}
\label{genheapsort}
\end{finalgo}
Jos kuitenkin kaarien painot ovat kokonaislukuja, $O(m + n \log n)$-rajaa voidaan parantaa:
Mikkel Thorup esitti vuonna 2003 keon, jonka poisto-operaatio käy ajassa $O(\log \log \min n)$ ja muut operaatiot vakioajassa \cite{Thorup03}. Jos kuitenkin kokonaislukupainot ovat väliltä $[0, N)$, poisto-operaatio voidaan suorittaa ajassa $O(\log \log \min \{ n, N \})$. Nyt selvästi haun aikavaativuus tällaisella keolla on $O(m + n \log \log \min \{n, N \})$.

\section{Kaikkien parien lyhimmät polut}

\bibliographystyle{babplain-lf}
\bibliography{refs}


% --- Appendices ---

% uncomment the following

% \newpage
% \appendix
% 
% \section{Esimerkkiliite}

\end{document}