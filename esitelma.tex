\documentclass{beamer}

\usepackage[utf8]{inputenc}
\usepackage[T1]{fontenc}
\usepackage[ruled,vlined,linesnumbered]{algorithm2e}
\usepackage{tikz}

\usepackage{verbatim}
\usetikzlibrary{arrows,shapes}

\SetAlFnt{\small}
\SetAlCapFnt{\large}
\SetAlCapNameFnt{\large}

%%% algorithm2e environment with "Algoritmi"-caption.
\newenvironment{finalgo}[1][htb]{
  \renewcommand{\algorithmcfname}{Algoritmi}
  \begin{algorithm}[#1]
}{\end{algorithm}}

%%% To be able to not numbering individual lines:
\let\oldnl\nl% Store \nl in \oldnl
\newcommand{\nonl}{\renewcommand{\nl}{\let\nl\oldnl}}

%%% argmin
\DeclareMathOperator*{\argmin}{arg\, min}

\title{{\rmfamily\scshape Lyhimpien polkujen hakualgoritmit ja -järjestelmät}}
\author{$\mathfrak{Rodion \, Efremov}$}
\date{}
\institute{Tietojenkäsittelytieteen laitos, Helsingin yliopisto}

\usetheme{Ilmenau}
\usecolortheme{beaver}
\usefonttheme[onlymath]{serif}

\begin{document}
\maketitle
% Declare layers
\pgfdeclarelayer{background}
\pgfsetlayers{background,main}

%%% BEGIN: Määritelmät
\begin{frame}
  \frametitle{Verkot}
  \begin{itemize}
    \item Suunnattu verkko on $G = (V, A)$, missä $V$ on solmujen joukko ja $A \subset V \times V$ on suunnattujen kaarien joukko.

    \item Suuntaamaton verkko $G = (V, E)$ voidaan aina simuloida suunnatulla verkolla $(V, A)$ laittamalla $A$:han kaaret $(u, v)$ ja $(v, u)$ jokaisella suuntaamattomalla kaarella $\{u, v\} \in E$.

    \item Jatkossa merkitsemme $n = |V|$ ja $m = |E|$.
  \end{itemize}
\end{frame}

\begin{frame}
  \frametitle{Verkot}
  \begin{itemize}
    \item $k$:n kaaren polku on $\gamma = \langle u_0, u_1, \dots, u_k \rangle$, missä kukin solmu esiintyy vain kerran ja jokaisella $i \in \{ 0, 1, \dots, k - 1 \}$ $(u_i, u_{i + 1}) \in A$.

    \item Painotettujen verkkojen kohdalla otaksumme kunkin kaaren $(u, v)$ painon $w(u, v)$ olevan \textbf{ei-negatiivinen}.
    
    \item Suuntamattomassa verkossa solmun $u$ ''aste'' on $d(u) = |\{ \{u, v\} \colon \{ u, v \} \in E \}|$. 
    
    \item Suunnatussa verkossa solmun $u$ ''sisäänaste'' (engl. \textit{indegree}) on $\textrm{deg}^-(u) = | \{ (v, u) \colon (v, u) \in A \} |$ ja
      ''ulosaste'' (engl. \textit{outdegree}) on $\textrm{deg}^+(u) = | \{ (u, v) \colon (u, v) \in A \} |$.
  \end{itemize}
\end{frame}
%%% END: Määritelmät

%%% BEGIN: BFS
\begin{frame}
\frametitle{Leveyssuuntainen haku}
\begin{itemize}
\item Löytää lyhimmän polun (yhden monesta mahdollisesta) painottamattomassa verkossa.
\item Toteutus vaatii vain jonon ja hajautustaulun.
\item Toimii ajassa $\mathcal{O}(n + m) \approx \sum_{i = 0}^N d_i$, missä $N$ on lyhimmän polun solmujen määrä ja $d$ keskiarvoinen solmun aste tai solmun ulosaste verkon tyypistä riippuen.
\end{itemize}
\end{frame}
%%% END: BFS

%%% BEGIN: BFS pseudocode
\begin{frame}
\begin{figure}[H]
  \includegraphics[width=\textwidth,keepaspectratio]{bfs}
\end{figure}
\end{frame}
%%% END: BFS pseudocode

\begin{frame}
Voiko leveyssuuntaisen haun nopeuttaa?
\end{frame}

\begin{frame}
Voi!
\end{frame}

\begin{frame}
\frametitle{Kaksisuuntainen leveyssuuntainen haku}
\begin{itemize}
  \item Aja kaksi hakuavaruutta: yksi normaalin tapaan lähtösolmusta, ja toinen ''takaperin'' maalisolmusta.
  \item Kun kaksi yllä mainittua hakuavaruutta kohtavat jossain ''keskellä'', rakennetaan lyhin polku.
  \item Aikavaativuus on $2\sum_{i = 0}^{\lceil N / 2\rceil} d^i$.
  \item Verkosta riippuen voi olla jopa \textasciitilde 1000 kertaa nopeampi kuin yksisuuntainen BFS.
\end{itemize}
\end{frame}

\begin{frame}
\begin{figure}[H]
  \includegraphics[width=\textwidth,height=\textheight,keepaspectratio]{bibfs}
\end{figure}
\end{frame}

\begin{frame}
Miten rakentaa polut lyhimpien polkujen puusta?
\end{frame}

\begin{frame}
Tarvitaan vain kuvaus $\pi$ (ja myös $\pi_{REV}$ mikäli polku oli haettu kaksisuuntaisella haulla).
\end{frame}

\begin{frame}
  \includegraphics[width=\textwidth,height=\textheight,keepaspectratio]{path}
\end{frame}

\begin{frame}
  \frametitle{Dijkstran algoritmi}
  \begin{itemize}
    \item Vuonna 1959 Edsger W. Dijkstra esitti kuuluisan polunhakualgoritminsa, joka toimii polynomisessa ajassa.
    \item FIFO jonon sijasta prioriteettijono; kutsutaan usein ''avoimeksi'' listaksi (engl. \textit{open set}).
    \item Hajautustauluun perustuva joukkorakenne; kutsutaan usein ''suljetuksi' listaksi (engl. \textit{closed set}).
    \item $g$-kuvaus, joka kuvaa kunkin saavutetun solmun toistaiseksi pienimpään etäisyyteen lähtösolmusta laskettuna.
    \item $\pi$-kuvaus, aivan kuten BFS:ssä (kuvaa solmun edeltäjäänsä lyhimpien polkujen puussa).
    \item Kun solmu poistetaan avoimesta listasta, sen $g$-arvo on optimaali.
  \end{itemize}
\end{frame}

\begin{frame}
  \includegraphics[width=\textwidth,height=\textheight,keepaspectratio]{dijkstra}
\end{frame}

\begin{frame}
  \frametitle{Dijkstran algoritmi}
  \begin{itemize}
    \item Dijkstran algoritmi voidaan mieltää BFS:n yleistykseksi painotetuissa verkoissa: samoin kuten BFS, Dijkstran algoritmi kasvattaa hakuavaruutensa ''kaikkiin suuntiin'' laajenevan pallon tavoin.
  \end{itemize}
\end{frame}

\begin{frame}
  \frametitle{A$\ast$ - haku}
  \begin{itemize}
    \item Pseudokoodi tasan sama kuten Dijkstran algoritmilla, paitsi että rivillä 3 $g(x)$ on korvattava $f(x)$:llä, jolle siis $f(x) = g(x) + h(x)$, missä $h(x)$ on optimistinen (eli aliarvioitu) kustannus solmusta $x$ maalisolmuun.
    
    \item Intuitio järjestelyn takana on se, että A$\ast$ ''tietää'' mihin suuntaan kannattaa lähteä kasvattamaan hakuavaruuden päästääkseen maalisolmuun nopeammin.
    
    \item Määrittämällä $h(u) = 0$ kaikilla $u \in V$, A$\ast$ palautuu Dijkstran algoritmiksi.
  \end{itemize}
\end{frame}

\begin{frame}
  \frametitle{Kaksisuuntainen painotettu haku}
  \begin{itemize}
    \item Myös A$\ast$ ja Dijkstran algoritmit voidaan kaksisuuntaista.
    \item Jos heuristiikkafunktio ei voida määritellä, kaksisuuntainen Dijkstran algoritmi on melkein aina parempit vaihtoehto suhteessa yksisuuntaiseen versioonsa.
    \item Mitä tulee A$\ast$:n kaksisuuntaistamiseen, algoritmi ei nopeudu erityisen paljon, sillä jo ei niin hyvä heuristiikkafunktio karsii hakuavaruuden melko hyvin.
  \end{itemize}
\end{frame}

\begin{frame}
  \frametitle{Kaksisuuntainen Dijkstran algoritmi}
  \includegraphics[width=\textwidth,height=\textheight,keepaspectratio]{update}
\end{frame}

\begin{frame}
  \frametitle{Kaksisuuntainen Dijkstran algoritmi}
  \includegraphics[width=\textwidth,height=\textheight,keepaspectratio]{expand}
\end{frame}

\begin{frame}
  \frametitle{Kaksisuuntainen Dijkstran algoritmi}
  \includegraphics[width=\textwidth,height=\textheight,keepaspectratio]{bidijkstra}
\end{frame}

\begin{frame}
  \frametitle{Kaksisuuntainen A$\ast$}
  Tasan sama kuin kaksisuuntainen Dijkstra, paitsi että operaation \textsc{Expand} rivillä 1 oleva $g(x)$ korvattava lausekkeella $f(x)$, jolle siis $f(x) = g(x) + h(x)$.
\end{frame}

\begin{frame}
  Mikä mahtaa olla tehokkain tapaa hakea polut?
\end{frame}

\begin{frame}
  \frametitle{Kaikkien parien lyhimmät polut}
  (1) Aja kaikkien-parit algoritmin, joka palauttaa nk. ''edeltäjämatriisin''.
\end{frame}

\begin{frame}
  \frametitle{Kaikkien parien lyhimmät polut}
  (2) Mikä tahansa $N$:n solmun polku voidaan rakentaa edellä mainitusta matriisista ajassa $\mathcal{O}(N)$!
\end{frame}

\begin{frame}
  \frametitle{Kaikkien parien lyhimmät polut}
  \begin{itemize}
    \item Floyd-Warshall toimii ajassa $\Theta(n^3)$.
    \item Jos $m = o(n^2)$, Johnsonin algoritmi Fibonacci-keolla on asymptoottisesti parempi: $\mathcal{O}(n^2 \log n + nm)$.
    \item Kumpikaan ei siis tarpeeksi tehokas prosessoimaan kokonaisen valtion tieverkkoa, sillä pelkästään solmuja on helposti yli 100000.
  \end{itemize}
\end{frame}

\begin{frame}
  \frametitle{Dijkstran algoritmi kaarivivuilla}
    Jaa kaikki solmut $V$ osituksiin $V_1, \dots, V_k$ siten, että
    \[
      \bigcup_{i = 1}^k V_i = V,
    \]
    ja $V_i \cap V_j = \emptyset$ kaikilla $i \neq j$.
\end{frame}

\begin{frame}
  \frametitle{Dijkstran algoritmi kaarivivuilla}
  $k$ osion ositus voidaan merkitä funktiolla $r \colon V \to \{ 1, 2, \dots, k\}$.
\end{frame}

\begin{frame}
  \frametitle{Dijkstran algoritmi kaarivivuilla}
  Jokaiselle kaarelle asetetaan $k$:n bitin bittivektorin (engl. \textit{arc-flag vector}, kaarivipuvektori); jos $i$des bitti on päällä, kaari on lyhimmällä polulla johonkin osion $V_i$ solmuun.
\end{frame}

\begin{frame}
  \frametitle{Dijkstran algoritmi kaarivivuilla}
  Osion $V_i$ ''rajasolmut'' ovat \[
    B_i = \{ u \in V_i \colon \exists (u, v) \in A \text{ siten että } r(v) \neq r(u) \}.
  \]
\end{frame}

\begin{frame}
  \frametitle{Dijkstran algoritmi kaarivivuilla}
  Jokaisen osion $V_i$ jokaisen rajasolmun $u \in B_i$ lähtien ajetaan ''takaperin'' tavallinen Dijkstra ja tuloksena syntyvässä lyhimpien polkujen puussa $T$ asetetaan jokaisen kaaren $(u, v) \in T$ $i$des vipu päälle.
\end{frame}

\begin{frame}
  \frametitle{Dijkstran algoritmi kaarivivuilla}
  Nyt tuloksena syntyvässä algoritmissa haettaessa polkua solmuun $t \in V$, voidaan karsia kaikki kaaret, joiden $r(t)$:s bitti ei ole päällä.
\end{frame}

\begin{frame}
  \frametitle{Dijkstran algoritmi kaarivivuilla}
  Voidaan myös kaksisuuntaista: jokaisella kaarella kaksi kaarivipuvektoria: yksi normaalia hakusuuntaa varten ja toinen käännettyä hakua varten.
\end{frame}

\begin{frame}
  \frametitle{Dijkstran algoritmi kaarivivuilla}
  Tekniikka voidaan nähdä tasapainoilevan tavallisen Dijkstran algoritmin ($k = 1, V_1 = V$) ja kaikkien parien algoritmin ($k = n$, jokainen solmu on osio) välillä.
\end{frame}

\begin{frame}
  \frametitle{Dijkstran algoritmi kaarivivuilla}
  Tutkijat raportoivat kaksisuuntaisen kaarivipu-Dijkstran olevan keskimäärin yli 500 kertaa nopeampi kuin tavallinen yksisuuntainen Dijkstra.
\end{frame}

\end{document}