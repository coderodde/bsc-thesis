\documentclass{beamer}

\usepackage[utf8]{inputenc}
\usepackage[T1]{fontenc}
\usepackage[ruled,vlined,linesnumbered]{algorithm2e}

\SetAlFnt{\small}
\SetAlCapFnt{\large}
\SetAlCapNameFnt{\large}

%%% algorithm2e environment with "Algoritmi"-caption.
\newenvironment{finalgo}[1][htb]{
  \renewcommand{\algorithmcfname}{Algoritmi}
  \begin{algorithm}[#1]
}{\end{algorithm}}

%%% To be able to not numbering individual lines:
\let\oldnl\nl% Store \nl in \oldnl
\newcommand{\nonl}{\renewcommand{\nl}{\let\nl\oldnl}}

%%% argmin
\DeclareMathOperator*{\argmin}{arg\, min}

\title{{\rmfamily\scshape Lyhimpien polkujen hakualgoritmit ja -järjestelmät}}
\author{$\mathfrak{Rodion \, Efremov}$}
\date{}
\institute{Tietojenkäsittelytieteen laitos, Helsingin yliopisto}

\usetheme{Ilmenau}
\usecolortheme{beaver}
\usefonttheme[onlymath]{serif}

\begin{document}
\maketitle

%%% BEGIN: Määritelmät
\begin{frame}
\frametitle{Verkot}
\begin{itemize}
\item Suunnattu verkko on $G = (V, A)$, missä $V$ on solmujen joukko ja $A \subset V \times V$ on suunnattujen kaarien joukko.

\item Suuntaamaton verkko $G = (V, E)$ voidaan aina simuloida suunnatulla verkolla $(V, A)$ laittamalla $A$:han kaaret $(u, v)$ ja $(v, u)$ jokaisella suuntaamattomalla kaarella $\{u, v\} \in E$.

\item Jatkossa merkitsemme $n = |V|$ ja $m = |E|$.
\end{itemize}
\end{frame}
%%% END: Määritelmät

%%% BEGIN: BFS
\begin{frame}
\frametitle{Leveyssuuntainen haku}
\begin{itemize}
\item Toteutus vaatii vain jonon ja hajautustaulun.
\item Toimii ajassa $\mathcal{O}(n + m) \approx \sum_{i = 0}^N d_i$, missä $N$ on lyhimmän polun solmujen määrä ja $d$ keskiarvoinen solmun aste.
\end{itemize}
\end{frame}
%%% END: BFS

%%% BEGIN: BFS pseudocode
\begin{frame}

\begin{finalgo}[H]
$Q, \pi = ( \{ s \}, \{ (s, \textbf{nil}) \} )$
$\text{OPEN}, \text{CLOSED}, g, \pi = (\{ s \}, \emptyset, \{  (s, 0) \}, \{ (s, \textbf{nil}) \})$ \\
\While{$|\text{OPEN}| > 0$}{
 $u = \underset{x \in \text{OPEN}}{\argmin}\, g(x)$ \\
 \If{$x \textbf{\emph{ is }} t$}{
   \KwRet \textsc{Traceback-Path$(t, \pi, \textbf{nil})$}\\
 }
 $\text{OPEN} = \text{OPEN} - \{ u \}$ \\
 $\text{CLOSED} = \text{CLOSED} \cup \{ x \}$ \\
 \nonl Jokaisella solmun $x$ lapsisolmulla $u$, tee... \\
 \For{$(x, u) \in G.A$}{
   \If{$u \in \text{\upshape CLOSED}$}{
     $\textbf{continue}$ \\
   }
   $g' = g(x) + w(x, u)$ \\
   \If{$u \not \in \text{\upshape OPEN}$}{
     $\text{OPEN} = \text{OPEN} \cup \{ u \}$ \\
     $g(u) = g'$ \\
     $\pi(u) = x$ \\
   }
 }
}
\nonl Ei $s, t$ -polkua verkossa $G$. \\
\KwRet $\langle \rangle$ \\
\caption{\textsc{Breadth-First-Search}$(G, s, t)$}
\end{finalgo}

\end{frame}
%%% END: BFS pseudocode

\end{document}