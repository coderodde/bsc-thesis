\documentclass[12pt]{article}

\usepackage[utf8]{inputenc}
\usepackage[T1]{fontenc}
\usepackage{amsfonts,amsmath,amssymb,amsthm,booktabs,color,enumitem,graphicx}

\title{Lyhimmät polut: kypsyysnäyte}
\author{Rodion ''rodde'' Efremov}

\begin{document}
\maketitle
\newpage

\noindent Haettaessa lyhimpiä polkuja verkossa on tärkeää olla tietoinen siitä, että tehokkain algoritmivalinta riippuu juuri annetusta verkosta ja mahdollisesti muun informaation saatavuudesta, kuten esimerkiksi heuristiikkafunktion. Toiset huomioon otettavat seikat ovat mm. solmujen määrä ja kaaritodennäköisyys (kaarien täyttöaste). Esimerkiksi, jos verkossa on noin 1000 solmua, kaikkien parien algoritmin ajo on realistinen ajatus, jolloin saadaan ''edeltäjämatriisi'', josta $N$ solmun polku voidaan rakentaa ajassa $\Theta(N)$. Selvästikin, on vaikea rakentaa lyhin polku tuon nopeammin, mutta tuhannen solmun suuruusluokka ei ole riittävä juuri koskaan mallintamaan kokonaisen valtion reittiverkostoa. Toinen ongelma on se, että käytännön reitinhakujärjestelmät käsittelevät dynaamisia verkkoja, jotka mallintavat liikenneinfrastruktuureja, ja joissa lyhin polku ei riipu pelkästään lähtö- ja maalisolmuista, vaan myös ajasta, jona matka aloitetaan. Käytännössä juuri tällaisissa järjestelmissä kaaripainojen lisäksi käytetään myös solmupainoja. Esimerkiksi, Pekka lähtee tietystä osoitteesta $A$ lähimpään bussipysäkkiin $B$, jossa hän odottaa ajan $t$ ennen kuin bussi saapuu. Nyt, reitinhakualgoritmi ei tallenna $B$:n prioriteetiksi vain $t' = w_{kävely}(A, B)$, vaan $t' + t$.


\end{document}