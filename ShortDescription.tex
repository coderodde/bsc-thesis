\documentclass[10pt]{article}

\usepackage[finnish]{babel}
\usepackage[utf8x]{inputenc}
\usepackage{amsmath}
\usepackage{amssymb}
\usepackage{amsfonts}
\usepackage{graphicx}
\usepackage[ruled,linesnumbered,noend]{algorithm2e}
\usepackage{empheq}
\usepackage{float}
\usepackage{enumitem}

\title{Polunhakualgoritmit ja -järjestelmät - aihealueen lyhyt kuvaus}
\author{Rodion ``rodde'' Efremov}

\begin{document}
 \maketitle
 
\noindent Vuonna 1959 Dijkstra esitti polynomisessa ajassa toimivan algoritmin \cite{Dijkstra59}, jolla voi etsiä lyhimmän polun kahden mielivaltaisen solmun välillä painotetuissa verkoissa. Algoritmi kasvattaa lyhimpien polkujen puun lähtösolmusta aina siihen asti kunnes puu kohtaa maalisolmun. Kesti noin yhdeksän vuotta kunnes ongelma sai uuden ratkaisun: $A\ast$ \cite{Hart68} käyttää optimistisen (etäisyyttä aliarvioivan) heuristiikkafunktion, jonka myöten algoritmi ``tietää'', mihin suuntaan kannattaa lähteä päästääkseen maalisolmuun, ja, täten, tarjoaa paremman tehokkuuden kuin Dijkstran algoritmi. $A\ast$:n heikkoutena suhteessa Dijkstran algoritmiin on kuitenkin juuri heuristiikkafunktion tarve, jota, ongelmasta riippuen, ei aina pystytä tyydyttämään. Lisäksi, $A\ast$ sai kilpailevan algoritmin: ironisesti, juuri Dijkstran algoritmin \textit{kaskisuuntaisesta} (engl. \textit{bidirectional}) versiosta, joka kasvattaa kaksi lyhimpien polkujen puuta molemmasta päätesolmusta kunnes kaksi puuta kohtaavat. Ellei oteta kantaa käytettävään prioriteettijonoon, yksisuuntainen haku tekee $\sum_{i = 0}^N d^i$ verran työtä siinä missä kaksisuuntainen tekee vaan $2\sum_{i = 0}^{N/2} d_i$, missä lyhimmän polun kaarien määrä on $N$ ja $d$ on verkon solmujen keskiarvoinen aste. Kaksisuuntaisuus on kannattava nopeutus aina kun heuristiikkafunktiota ei ole mahdollista johtaa, eli Dijkstran algoritmi ja myös leveyssuuntainen haku hyötyvät ko. järjestelystä. Kaksisuuntainen $A\ast$ ei kuitenkaan välttämättä aina tekee vähemmän laskentatyötä kuin yksisuuntainen versionsa, sillä on mahdollista että kaksisuuntaisen version kaksi hakupalloa ohittavat esteen verkossa eri puolilta, näin tehden enemmän työtä kuin yksisuuntainen $A\ast$.
\vspace{0.5em}

\noindent Kaksisuuntaisuus ei kuitenkaan ole ainoa kirjallisuudessa esiintyvä nopeutustekniikka: vuonna 2003 Meyer et al. esittivät $\Delta$-stepping algoritmin \cite{Meyer98}, joka $2^{19}$ solmun harvalla (engl. \textit{sparse}) verkolla oli noin 3.1 kertaa nopeampi kuin optimoitu Dijkstran algoritmin, ja 16:n suorittimen koneessa rinnakkainen $\Delta$-stepping oli 9.2 kertaa nopeampi kuin peräkkäinen versionsa. Toinen huomionarvoinen nopeutustekniikka on vuonna 2007 esitetty \textit{Dijkstra with arc-flags}\cite{Mohring07}, jonka ideana on osittaa verkon solmut $V$ komponentteihin $V_1, \dots, V_n$ siten, että $\cup_{i = 1}^n V_i = V$ ja $V_i \cap V_j = \varnothing$ kaikilla $i \neq j$, ja jokaiselle suunnatulle kaarelle liittää bittivektorin, jonka $i$:s bitti on päällä jos ja vain jos kyseinen kaari on lyhimmällä polulla johonkin osion $V_i$ solmuun. Ne verkon solmut $u$, joista lähtee edes yksi kaari solmuun $v$ siten, että $u$ ja $v$ eivät ole samassa osiossa sanotaan \textit{rajasolmuiksi}, ja verkon esikäsittelyssä järjestelmä rakentaa lyhimpien polkujen puut jokaisesta rajasolmusta, jolloin on toivottava minimoida rajasolmujen määrää: $kd$-puualgoritmi on helppo toteuttaa, mutta ei onnistu minimoimaan rajasolmuja niin hyvin, kuin METIS \cite{Karypis98}, joka ei edes tarvitse solmujen koordinaatteja kuin $kd$-puu. Yhdessä kaksisuuntaisuuden ja kahden tason METIS-osituksen kanssa -- jossa myös jokainen osio on osioitu pienempiin solmujoukkoihin -- Möhring et al. raportoivat keskimäärin yli 500 kertaa nopeammat polunhakuoperaatiot verrattuna alkuperäiseen Dijkstran algoritmiin.

\bibliographystyle{plain}
\bibliography{refs}

\end{document}